\documentclass[12pt]{article}
\usepackage{graphicx} % This lets you include figures
\usepackage[rightcaption]{sidecap}
\usepackage{subcaption}
\usepackage{wrapfig}
\usepackage{float}
\usepackage[spanish]{babel}
\languageshorthands{spanish}
\usepackage{imakeidx}
\usepackage{cite}
\makeindex
\usepackage{hyperref}

\title{Manual Técnico \\ Carta3}
\author{Luis A. Alvarado}
\date{\today}
\usepackage{metalogo}
\begin{document}
\maketitle{}

\tableofcontents

\clearpage
\newpage

\section{Paquetes}
\subsection{amsmath}
Proporciona un conjunto de opciones para mostrar ecuaciones. \cite{amsmath}
\subsection{amsfonts}
Colección amplia de fuentes para usar en matemáticas. \cite{amsfonts}
\subsection{amssymb}
Ofrece toneladas de símbolos matemáticos, como flechas, operadores, caracteres especiales, figuras geométricas, etc. \cite{amssymb}
\subsection{graphicx}
Se basa en el paquete \textit{graphics} y proporciona una interfaz clave-valor para argumentos opcionales del comando \textit{\textbackslash includegraphics}. Esta interfaz proporciona funciones que van mucho más allá de lo que ofrece el paquete de \textit{graphics} por sí solo. \cite{graphicx}
\subsection{pdfpages}
Simplifica la inclusión de documentos PDF externos de varias páginas en documentos \LaTeX.\cite{pdfpages}
\subsection{setspace}
Agrega soporte para establecer el espacio entre líneas en un documento. Las opciones de paquete incluyen espacio simple, espacio medio y espacio doble. \cite{setspace}
\subsection{xltxtra}
Se utiliza para proporcionar una serie de funciones que son útiles para la composición tipográfica de documentos con \XeLaTeX. \cite{xltxtra}
\subsection{enumitem}
Permite el control sobre el diseño de los tres entornos de lista básicos: \textit{enumerate}, \textit{itemize} y \textit{description}.\cite{enumitem}
\subsection{fixltx2e}
Este paquete no hace más que emitir una advertencia en las versiones actuales de \LaTeX. En versiones anteriores, se usaba para distribuir correcciones de errores y mejoras en el kernel de \LaTeX. \cite{fixltx2e}
\subsection{xifthen}
Este paquete amplía el paquete \textit{ifthen} al implementar nuevos comandos para ir dentro del primer argumento de \textit{\textbackslash ifthenelse} para probar si una cadena es nula o no, si un comando está definido o es equivalente a otro. El paquete también permite el uso de expresiones complejas tal como las introdujo el paquete \textit{calc}, junto con la capacidad de definir nuevos comandos para manejar pruebas complejas.\cite{xifthen}
\subsection{xargs}
Agrega versiones extendidas de \textit{\textbackslash newcommand} y comandos \LaTeX relacionados, que permiten una definición fácil y robusta de macros con muchos argumentos opcionales, usando una sintaxis clara.\cite{xargs}
\subsection{booktabs}
Mejora la calidad de las tablas en LATEX, brindando comandos adicionales y optimización entre bastidores.\cite{booktabs}
\subsection{multirow}
Agrega mucha flexibilidad, incluida una opción para especificar una entrada en el ancho "natural" de su texto. Cree celdas tabulares que abarquen varias filas.\cite{multirow}

\newpage
\bibliographystyle{unsrt}
\bibliography{bibliografia}
\end{document}